\section{Ähnliche Spiele}
Es gibt bereits diverse ``Player vs Player''-Brawlspiele, also Spiele, bei denen die Spieler gegeneinander kämpfen. Im Folgenden werden ein paar davon erläutert:

\subsection{Super Smash Bros (SSB)}
Super Smash Bros (SSB) sind eine Reihe von Videospielen, bei denen man versucht seine Gegner aus der Kampfarena zu katapultieren.
Diese sind von Nintendo entwickelt und beinhalten die bekanntesten Charakteren des Unternehmens.
Figuren wie Super Mario oder Sonic bekämpfen sich in einer Arena mit dem Ziel sich gegenseitig
von einer Plattform zu stoßen.
Was jedoch fehlt, ist die Plattformunabhängigkeit, da das Spiel nur auf Nintendo-Systemen funktioniert.
Außerdem besteht eine Limitierung in der Auswahl von Spielumgebungen.

\subsection{Brawlhalla} \cite{brawlhalla}
Brawlhalla ist ein von der Firma Blue Mammoth entwickeltes 2D-Kampfspiel, und wurde für alle gängigen Betriebssysteme entwickelt.
Auch wie in ScribbleFight ist es das Ziel, den Gegner von einer Plattform zu stoßen.

Ein Nachteil hierbei ist, dass sich der Benutzer vor dem Spielen einen Account erstellen
und dann einen Download abschließen muss. Dazu kommt noch, dass die Spielumgebung nie beeinflusst werden kann.

\subsection{Stick Fight: The Game} \cite{stickfight}
In diesem Spiel kämpfen die Spieler und Spielerinnen als Strichmännchen gegeneinander, die durch eine Ragdoll-Engine gesteuert werden, eine Engine, die das Bewegungsverhalten von menschlichen Körpern simuliert.
Auch wie bei schon bei vorher erwähnten Spielen, kann der Benutzer aber nicht einfach plattformunabhängig im Browser gegeneinander antreten.
Für die Spielekonsole Nintendo Switch zum Beispiel, gibt es das Spiel auch nicht. Hinzu kommt, dass auch die Umgebung nicht frei erstellt werden kann.

\section{Ist-Zustand}
Die Idee hinter ScribbleFight ist einzigartig. Deshalb muss das Produkt von null auf umgesetzt werden. Als Hilfe wurden jedoch Bibliotheken verwendet, wie zum Beispiel p5.js (\ref{subsection:p5js}) für den Client des Spiels, oder openCV für die Bilderkennung (\ref{maai:maperkennung:head}).
Mathematische Modelle, welche in der ``Python'' Bibliothek ``Stable Baselines3'' bereits vorgefertigt sind, wurden
als Hilfestellung zum Trainieren der Künstlichen Intelligenz verwendet.