
\section{JavaScript [R]}
Für mich war es eine leichte Entscheidung JavaScript zu verwenden, da die Programmiersprache ideal für den Browser geeignet ist, und man damit nicht nur im Frontend, sondern auch im Backend programmieren kann.
\setauthor{Rafetseder Tobias}
\subsection{p5.js / p5.play [R]}
p5.js ist eine open-source JavaScript Library, die für Kreation von Spielen genutzt wird. 
p5.play ist eine Library für p5.js, mit der man visuelle Objekte managen kann. Außerdem beinhaltet es Features wie Animation-support, 
Kollisionserkennung, sowie aber auch Funktionen für Mouse und Tastatur Interaktionen.
Es ist wichtig sich im Hinterkopf zu behalten, dass p5.play für barrierefreies und simples Programmieren gedacht ist, nicht für perfomantes.
Es ist keine eigene Engine, und unterstütz auch keine 3D-Spiele.

p5.js ähnelt sich sehr stark mit Processing, eine Programmiersprache die man sich wie ein stark vereinfachte Version von Java vorstellen kann.
Der Unterschied liegt darin, dass Java eine Umgebung, basierend auf der Java Programmiersprache ist, während p5.js eine Bibliothek, basierend auf der JavaScript Programmiersprache ist.
Processing ist dafür geeignet, lokale Applikationen zu bauen, hingegen dazu kann p5.js nur im Browser ausgeführt werden.

p5.js ist also kurzgesagt ein direkter JavaScript Port für die Processing Programmiersprache. 

Vorteile von p5:
\begin{compactitem}
    \item Man kann interaktive Programme entwickeln, die in jedem modernen Browser funktionieren (plattformunabhängig)
    \item Das Programm ist nicht nur lokal auf dem eigenen Gerät, was das Teilen sehr viel leichter macht
    \item Man hat die Option den p5.js Editor im Web zu verwenden: Überhaupt kein Aufwand, um loszuprogrammieren
\end{compactitem}

Nachteile von p5:
\begin{compactitem}
    \item Ist schneller beim Pixel manipulieren
    \item Man kann Java Bibliotheken verwenden
\end{compactitem}

\subsection {node.js [R]}
\subsection{Socket IO [R]}

\section{Cloud Computing [R]}
\subsection{Docker [R]}
\subsection{Kubernetes [R]}


\section{Python [H]}
\setauthor{Himmetsberger Jonas}

\subsection{Flask [H]}

\subsection{OpenCV2 [H]}

\subsection{PIL [H]}

\subsection{TensorFlow und Keras [H]}

\subsection{OpenAI Gym [H]}
