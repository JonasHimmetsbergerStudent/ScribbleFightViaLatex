\section{Projektanlass}
Die Möglichkeit seiner Kreativität freien Lauf zu lassen ist bei den meisten populären
Online-Spielen sehr eingeschränkt, da man wenig Einfluss auf die Spielumgebung hat.
Unsere Arbeit soll diesem Problem entgegenwirken. Der Spieler kann selbst entscheiden,
wie die 2D-Spielumgebung auszusehen hat, indem er diese auf ein Blatt Papier zeichnet,
welche dann via Bilderkennung als spielbare Welt aufbereitet wird.

\section{Ziele}
Bis zum Abgabetermin sollen folgende Ziele erreicht werden:
\begin{compactitem}
    \item Das Spiel soll als Browserspiel funktionsfähig sein.
    \item Für die Benutzer soll es möglich sein ihre eigenen Kampfumgebungen zu erschaffen.
    \item Das Spiel soll als online-Multiplayer "Player versus Player"-Spiel funktionieren.
    \item Ein eigener Modus, in welchem der Spieler als Singleplayer gegen eine funktionsfähige KI antreten kann, soll umgesetzt werden.
\end{compactitem}

\section{Aufgabenverteilung}
Dadurch, dass die Diplomarbeit drei mitwirkende Schüler hat, wurde das Thema in drei ähnlich
anspruchsvolle Teile unterteilt. Diese sind:
\begin{compactenum}
    \item Gamephysics, Hitregistration, Collisiondetection, Regeln und Spielablauf des Spiels, Deployment des Spiels
    \item Gamedesign, Sounddesign, Frontend, Animationen, Multiuser-Fähigkeit (QR-Code, Lobby, ...)
    \item Objekterkennung von Blatt Papier (Aufbereitung der Spielumgebung), Trainings-Bot KI, Forschung (Reinforcement learning für KI)
\end{compactenum}
Im folgenden werden die Themen genauer erläutert:

\subsection{Web-Game und Deployment [R]}

\subsection{Gamedesign und Lobby-System [B]}

\subsection{Aufbereitung der Spielumgebung und Künstliche Intelligenz [H]}

Im Laufe der Diplomarbeit und der damit zusammenhängenden Forschung änderte sich oft die
Vorstellung darüber, wie das Endprodukt auszusehen hat.


\section{Dokumente}
