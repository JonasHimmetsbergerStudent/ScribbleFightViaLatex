\section{Projektanlass}
Die Möglichkeit seiner Kreativität freien Lauf zu lassen ist bei den meisten populären
Online-Spielen sehr eingeschränkt, da man wenig Einfluss auf die Spielumgebung hat.
Diese Arbeit soll diesem Problem entgegenwirken. Der Spieler kann selbst entscheiden,
wie die 2D-Spielumgebung auszusehen hat, indem er diese auf ein Blatt Papier zeichnet,
welche dann via Bilderkennung als spielbare Welt aufbereitet wird.

\section{Ziele}
\begin{compactitem}
    \item Das Spiel soll im Browser spielbar sein.
    \item Für die Benutzer soll es möglich sein eigene Kampfumgebungen zu erschaffen.
    \item Das Spiel soll online als Multiplayer ``Player versus Player''-Spiel funktionieren.
    \item Ein eigener Modus, in welchem der Spieler als Singleplayer gegen eine funktionsfähige KI antreten kann.
\end{compactitem}

\section{Aufgabenverteilung}
Dadurch, dass die Diplomarbeit drei mitwirkende Schüler hat, wurde das Thema in drei ähnlich
anspruchsvolle Teile unterteilt. Diese werden im Folgenden genauer erläutert:
% \begin{compactenum}
%     \item Gamephysics, Hitregistration, Collisiondetection, Regeln und Spielablauf des Spiels, Deployment des Spiels
%     \item Gamedesign, Sounddesign, Frontend, Animationen, Multiuser-Fähigkeit (QR-Code, Lobby, ...)
%     \item Objekterkennung von Blatt Papier (Aufbereitung der Spielumgebung), Trainings-Bot KI, Forschung (Reinforcement learning für KI)
% \end{compactenum}

\subsection{Web-Game und Deployment [R]}
Die wichtigsten Punkte, die im Bezug auf das Web-Game umzusetzen waren, sind:
\begin{compactitem}
    \item Am Client:
    \begin{enumerate}
        \item Die Spielphysik, also wie sich Spieler und Objekte verhalten
        \item Die Collision-Detection von Spielern mit der Umgebung und mit Objekten
        \item Die Hitregistration, falls ein Spieler von etwas getroffen wurde
        \item Ab wann ist das Spiel zu Ende, beziehungsweise wann hat jemand gewonnen
    \end{enumerate}
    \item Am Server:
    \begin{enumerate}
        \item Übertragung von Positionen der Spieler
        \item Item-Erstellung bei allen Clients zum gleichen Zeitpunkt
        \item Wenn jemand eine Attacke ausführt soll diese bei allen Clients erstellt werden
        \item Wenn Items aufgesammelt werden, sollen diese bei allen verbundenen Spielern verschwinden
        \item Wenn eine Attacke jemanden trifft, soll diese bei allen verbundenen Spielern gelöscht werden
    \end{enumerate}

\end{compactitem}
Das Deployment des Projekts in die Leocloud, ein Cloud-System der HTL-Leonding, ist mittels Kubernetes erfolgt.


\subsection{Gamedesign und Lobby-System [B]}

\subsection{Aufbereitung der Spielumgebung und Künstliche Intelligenz [H]}
Die Aufbereitung der Spielumgebung sollte folgende Funktionen erfüllen:
\begin{compactitem}
    \item Erkennung des Blatt Papiers in einer Live-View
    \item Aufnahme soll in eine digitale Spielumgebung umgewandelt werden
\end{compactitem}

Folgende Forderungen waren an die KI gestellt:
\begin{compactitem}
    \item Die Künstliche Intelligenz soll auf einem herausfordernden Niveau agieren
    \item Im Zuge der Forschung sollten zwei gängige Lernalgorithmen verglichen werden
\end{compactitem}


Im Laufe der Diplomarbeit und der damit zusammenhängenden Forschung änderte sich einige Male die
Vorstellung darüber, wie das Endprodukt auszusehen hat.


\section{Dokumente}
