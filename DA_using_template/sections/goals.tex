\section{Projektanlass}
Die Möglichkeit seiner Kreativität freien Lauf zu lassen ist bei den meisten populären
Online-Spielen sehr eingeschränkt, da man wenig Einfluss auf die Spielumgebung hat.
Unsere Arbeit soll diesem Problem entgegenwirken. Der Spieler kann selbst entscheiden,
wie die 2D-Spielumgebung auszusehen hat, indem er diese auf ein Blatt Papier zeichnet,
welche dann via Bilderkennung als spielbare Welt aufbereitet wird.

\section{Ziele}
Bis zum Abgabetermin sollen folgende Ziele erreicht werden:
\begin{compactitem}
    \item Das Spiel soll als Browserspiel funktionsfähig sein.
    \item Für die Benutzer soll es möglich sein ihre eigenen Kampfumgebungen zu erschaffen.
    \item Das Spiel soll als online-Multiplayer "Player versus Player"-Spiel funktionieren.
    \item Ein eigener Modus, in welchem der Spieler als Singleplayer gegen eine funktionsfähige KI antreten kann, soll umgesetzt werden.
\end{compactitem}

\section{Aufgabenverteilung}
Dadurch, dass die Diplomarbeit drei mitwirkende Schüler hat, wurde das Thema in drei ähnlich
anspruchsvolle Teile unterteilt. Diese werden im Folgenden genauer erläutert:
% \begin{compactenum}
%     \item Gamephysics, Hitregistration, Collisiondetection, Regeln und Spielablauf des Spiels, Deployment des Spiels
%     \item Gamedesign, Sounddesign, Frontend, Animationen, Multiuser-Fähigkeit (QR-Code, Lobby, ...)
%     \item Objekterkennung von Blatt Papier (Aufbereitung der Spielumgebung), Trainings-Bot KI, Forschung (Reinforcement learning für KI)
% \end{compactenum}

\subsection{Web-Game und Deployment [R]}
Die wichtigsten Punkte, die im Bezug auf das Web-Game umzusetzen zu waren, sind:
\begin{compactitem}
    \item Die Spielphysik, also wie sich Spieler und Objekte verhalten
    \item Die Collisiondetection von Spielern mit der Umgebung und mit Objekten
    \item Die Hitregistration, falls ein Spieler von etwas getroffen wurde
    \item Ab wann ist das Spiel zu Ende, beziehungsweise wann hat jemand gewonnen
\end{compactitem} 
Das Deployment des Projekts in die Leocloud, ein Cloud-System der HTL-Leonding, soll mittels Kubernetes erfolgen.


\subsection{Gamedesign und Lobby-System [B]}

\subsection{Aufbereitung der Spielumgebung und Künstliche Intelligenz [H]}
Die Aufbereitung der Spielumgebung sollte folgende Funktionen erfüllen:
\begin{compactitem}
    \item Erkennung der Konturen in einer Live-View
    \item Aufnahme soll in brauchbare Daten umgewandelt werden
\end{compactitem}

Folgende Forderungen waren an die KI gestellt: 
\begin{compactitem}
    \item Die KI soll auf einem herausfordernden Niveau agieren
    \item Im Zuge der Forschung sollte ein Vergleich zwischen brauchbaren Lernalgorithmen gemacht werden
\end{compactitem}


Im Laufe der Diplomarbeit und der damit zusammenhängenden Forschung änderte sich oft die
Vorstellung darüber, wie das Endprodukt auszusehen hat.


\section{Dokumente}
