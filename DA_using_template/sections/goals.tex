\section{Ziele}
Bis zum Abgabetermin sollen folgende Ziele erreicht werden:
\begin{compactitem}
    \item Das Spiel soll als Browserspiel funktionsfähig sein.
    \item Für die Benutzer soll es möglich sein ihre eigenen Kampfumgebungen zu erschaffen.
    \item Das Spiel soll als online-Multiuser "Player versus Player"-Spiel funktionieren.
    \item Eigener Modus, in welchem der Spieler als Singleplayer gegen eine funktionsfähige KI antreten kann.
\end{compactitem}

\section{Projektanlass}
Die Möglichkeit seiner Kreativität freien Lauf zu lassen ist bei den meisten populären
Online-Spielen sehr eingeschränkt, da man wenig Einfluss auf die Spielumgebung hat.
Unsere Arbeit soll diesem Problem entgegenwirken. Der Spieler kann selbst entscheiden,
wie die Spielumgebung auszusehen hat, indem er diese auf ein Blatt Papier zeichnet,
welche dann via Bilderkennung als spielbare Welt aufbereitet wird.


\section{Aufgabenverteilung}
Dadurch, dass die Diplomarbeit drei mitarbeitende Schüler hat wurde das Thema in drei ähnlich
anspruchsvolle Teile unterteilt. Diese sind:
\begin{compactitem}
    \item Gamephysics, Hitregistration, Collisiondetection, Regeln und Spielablauf
    \item Gamedesign, Sounddesign, Frontend, Animationen, Multiuser-Fähigkeit (QR-Code, Lobby, ...)
    \item Objekterkennung von Blatt Papier, Trainings-Bot KI, Forschung (Reinforcement learning für KI)
\end{compactitem}

\subsection{Aufgaben von Rafetseder Tobias}

\subsection{Aufgaben von Weinzierl Ben}

\subsection{Aufgaben von Himmetsberger Jonas}
\subsubsection{Objekterkennung von Blatt Papier}
\subsubsection{Künstliche Intelligenz}
Im Laufe der Diplomarbeit und der damit zusammenhängenden Forschung änderte sich oft die
Vorstellung darüber, wie das Endprodukt auszusehen hat.


\section{Dokumente}
\section{Meilensteine}
