\begin{spacing}{1}
  \chapter*{Abstract}
\end{spacing}
\begin{wrapfigure}{r}{0.3\textwidth}
  \begin{center}
    \includegraphics[width=0.2\textwidth]{pics/question_mark.png}
  \end{center}
\end{wrapfigure}
The project idea is to create a game, where you and your friends can join a lobby and make pictures of the environment you want to play on.
Next you vote for the picture that is the best out of all of them, and you can start playing on that map against each other. You have to shoot your friends with little projectiles, but can also pick up various items and use them.
\lipsum[6]
\newpage
\begin{spacing}{1}
  \chapter*{Zusammenfassung}
\end{spacing}
\begin{wrapfigure}{r}{0.3\textwidth}
  \begin{center}
    \includegraphics[width=0.2\textwidth]{pics/question_mark.png}
  \end{center}
\end{wrapfigure}
Scribble Fight ein plattformunabhängiges Spiel, dass es sich zur Aufgabe gemacht hat, dem Spieler oder der Spielerin so viel Freiraum wie möglich zu geben. 
Die Spieler können ihre Spielumgebung in der realen Welt zeichnen, und diese mithilfe der im Spiel integrierten Kamera, welche Open Computer Vision verwendet, digitalisieren.
Um mit seinen Freunden oder Freundinnen spielen zu können, wird eine Lobby erstellt und mit einem vierstelligen Code kann man dieser beitreten.
Nachdem dort die Spielumgebung ausgewählt wurde, können die Spieler gegeneinander antreten. 
Das Ziel ist, die Wettstreiter mithilfe von kleinen Projektilen abzuschießen, und dadurch diese von der vorher gezeichneten Umgebung zu werfen. 
Sind alle Mitspieler oder Mitspielerinnen drei mal hinuntergefallen, und der Spieler oder der Spielerin selbst ist noch am Leben, hat dieser oder diese gewonnen.
Im Zuge des Projektverlaufs wurde auch ein Programm entwickelt, das menschliches Spielverhalten erlernen sollte.
Das Projektziel ist es, den Spielern eine kurzweilige, aber spannende Unterhaltung zu bieten.
\emph{Bitte auf keinen Fall mit der Zusammenfassung verwechseln, die den Abschluss der Arbeit bildet!}
\lipsum[6]
