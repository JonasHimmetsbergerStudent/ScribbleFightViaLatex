\begin{spacing}{1}
  \chapter*{Abstract}
\end{spacing}
\begin{wrapfigure}{r}{0.3\textwidth}
  \begin{center}
    \includegraphics[width=0.2\textwidth]{pics/question_mark.png}
  \end{center}
\end{wrapfigure}
Scribble Fight is a platform independent game that aims to give the player as much freedom as possible. 
Players can draw their environment in the real world and digitize it using the game's built-in camera, which uses Open Computer Vision.
In order to play with friends, a lobby can be created and a four-digit code is used to join it.
After selecting the game environment, players can compete against each other. 
The goal is to shoot the competitors with the help of small projectiles, thereby throwing them off the previously drawn environment. 
If all other players have fallen down three times and the player is still alive, he or she has won.
In the course of the project, a program was also developed to learn human gaming behavior.
The aim of the project is to provide the players with short-lasting but exciting entertainment.
\lipsum[6]
\newpage
\begin{spacing}{1}
  \chapter*{Zusammenfassung}
\end{spacing}
\begin{wrapfigure}{r}{0.3\textwidth}
  \begin{center}
    \includegraphics[width=0.2\textwidth]{pics/question_mark.png}
  \end{center}
\end{wrapfigure}
Scribble Fight ein plattformunabhängiges Spiel, dass es sich zur Aufgabe gemacht hat, dem Spieler oder der Spielerin so viel Freiraum wie möglich zu geben. 
Die Spieler können ihre Spielumgebung in der realen Welt zeichnen, und diese mithilfe der im Spiel integrierten Kamera, welche Open Computer Vision verwendet, digitalisieren.
Um mit seinen Freunden oder Freundinnen spielen zu können, wird eine Lobby erstellt und mit einem vierstelligen Code kann man dieser beitreten.
Nachdem dort die Spielumgebung ausgewählt wurde, können die Spieler gegeneinander antreten. 
Das Ziel ist, die Wettstreiter mithilfe von kleinen Projektilen abzuschießen, und dadurch diese von der vorher gezeichneten Umgebung zu werfen. 
Sind alle Mitspieler oder Mitspielerinnen drei mal hinuntergefallen, und der Spieler oder der Spielerin selbst ist noch am Leben, hat dieser oder diese gewonnen.
Im Zuge des Projektverlaufs wurde auch ein Programm entwickelt, das menschliches Spielverhalten erlernen sollte.
Das Projektziel ist es, den Spielern eine kurzweilige, aber spannende Unterhaltung zu bieten.
\emph{Bitte auf keinen Fall mit der Zusammenfassung verwechseln, die den Abschluss der Arbeit bildet!}
\lipsum[6]
