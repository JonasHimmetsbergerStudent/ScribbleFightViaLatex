\begin{spacing}{1}
  \chapter*{Abstract}
\end{spacing}
\begin{wrapfigure}{r}{0.3\textwidth}
  \begin{center}
    \includegraphics[width=0.2\textwidth]{pics/question_mark.png}
  \end{center}
\end{wrapfigure}
``ScribbleFight'' is a platform-independent game that allows players to give free rein to their creativity.
They can draw their game environment in the real world and digitize it using the implemented image recognition, which uses Open Computer Vision.
In order to play with friends, a lobby is created and a four-digit code is used to join it.
After selecting the game environment, players can compete against each other.
The goal is to shoot the competitors with the help of small projectiles and thereby throw them from the previously drawn environment.
Once all opponents have fallen down three times, the last survivor wins.
In the course of the project, a program was also developed to learn human game behavior.
The project goal is to provide players with an exciting and dynamic entertainment experience.
% \lipsum[6]
\newpage
\begin{spacing}{1}
  \chapter*{Zusammenfassung}
\end{spacing}
\begin{wrapfigure}{r}{0.3\textwidth}
  \begin{center}
    \includegraphics[width=0.2\textwidth]{pics/question_mark.png}
  \end{center}
\end{wrapfigure}
``ScribbleFight'' ist ein plattformunabhängiges Spiel, welches den Spielern und Spielerinnen erlaubt ihre Kreativität freien Lauf zu lassen.
Sie können ihre Spielumgebung in der realen Welt zeichnen und diese mithilfe der implementierten Bilderkennung, welche Open Computer Vision verwendet, digitalisieren.
Um mit seinen Freunden oder Freundinnen spielen zu können, wird eine Lobby erstellt und mit einem vierstelligen Code kann dieser beigetreten werden.
Nachdem die Spielumgebung ausgewählt wurde, können die Spieler gegeneinander antreten.
Das Ziel ist die Wettstreiter mithilfe von kleinen Projektilen abzuschießen und dadurch diese von der vorher gezeichneten Umgebung zu werfen.
Sobald alle Gegner dreimal hinuntergefallen sind, hat der oder die letzte Überlebende gewonnen.
Im Zuge des Projektverlaufs wurde auch ein Programm entwickelt, das menschliches Spielverhalten erlernen sollte.
Das Projektziel ist es, den Spielern eine spannende und dynamische Unterhaltung zu bieten.
% \emph{Bitte auf keinen Fall mit der Zusammenfassung verwechseln, die den Abschluss der Arbeit bildet!}
% \lipsum[6]
