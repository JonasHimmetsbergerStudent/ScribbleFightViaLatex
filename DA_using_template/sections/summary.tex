\section{Meilensteine}
\begin{table}[H]
    \centering
    \begin{tabular}{|l|l|l|}
        \hline
        \multicolumn{1}{|c|}{\textit{}} & \textbf{Meilenstein}               & \textbf{Berichtszeitpunkt} \\ \hline
        10.07.2021                      & \begin{tabular}[c]{@{}l@{}}Recherche und Findung der\\ Technologien {[}H,R,W{]}\end{tabular}          & 10.07.2021                 \\ \hline
        25.07.2021                      & \begin{tabular}[c]{@{}l@{}}Map Erkennung von Blatt Papier \\ und Implementierung in Spiel {[}H{]}\end{tabular}          & 25.07.2021                 \\ \hline
        31.07.2021                      & Gamephysics und Gamedesign {[}R{]} & 01.09.2021                 \\ \hline
        30.09.2021                      & Mulitplayer-Funktion {[}R{]}       & 30.10.2021                 \\ \hline
        14.10.2021                      & Umsetzung der KI {[}H{]}           & 14.11.2021                 \\ \hline
        30.09.2021                      & Lobbysystem {[}W{]}                & 30.03.2022                 \\ \hline
        31.12.2021                      & Zusammenführung {[}W{]}            & 30.03.2022                 \\ \hline
    \end{tabular}
\end{table}

\section{Gelerntes}
Während des Projektverlaufs haben wir eine Menge gelernt. Und das nicht nur im technischen
Bereich, sondern auch in Dingen wie Planung, Team Arbeit, etc. \\
So war es uns möglich, Wissen im Bereich Künstliche Intelligenz zu sammeln und dieses Wissen auch anzuwenden.
Auch zum Thema Spielentwicklung wurde einiges gelernt, zum Beispiel was am Server und was am Client passieren soll, und wie diese überhaupt miteinander kommunizieren.
Weiteres wurde auch einiges an Wissen im Bereich Docker und Kubernetes angeeignet, womit auch das Deployment umgesetzt werden konnte.

Andererseits hat es sich bei uns bestätigt, dass es bei Software-Projekten oft zu Verzögerungen kommen kann.
So wurde oftmals der Aufwand unterschätzt, der in Komponenten des Software-Projekts von Nöten war.
Auch das Zusammenarbeiten war in der Theorie einfacher, als in der Praxis. So war das Kommunizieren, welche Daten das Web-Spiel abrufbereit haben muss, damit die Künstliche Intelligenz daraus lernen kann, oft schwierig.

Auch lernten wir besser das Umgehen mit Versionsverwaltungen, wie in unserem Fall Git, damit wir immer auf einem gemeinsamen Stand waren.

\section{Was würden wir anders machen?}
Was sich während dem Projektverlauf eindeutig aufgezeigt hat, war, dass die Eigenschaft von Java-Script, auf Variablen zuzugreifen zu können ohne sie initialisieren zu müssen, ein großes Problem geworden ist.
Vor allem dann, wenn die Künstliche Intelligenz Tage lang trainiert hat, kam es zu unvorhersehbaren Fehlern, was den Trainingsvorgang verzögert hat.
Auch wenn p5.js ein Framework ist, mit dem das Programmieren einfach bleibt, ist es nicht sonderlich gut für komplexere Spiele geeignet.

Die Forschung zum Thema Künstliche Intelligenz ergab, dass die verwendeten, vorgefertigete Algorithmen nicht perfekt auf den Anwendungsfall abgestimmt sind.
Alle Parameter dieser mathematischen Modelle sind bereits vordefiniert und wurden nicht abgeändert.
Sogenanntes Hpyerparamter ``Fine-tuning'' würde diesem Problem entgegenwirken.

Damit stark zusammenhängend ist die Komplexität, der Aufgabe, welche die KI erlernen hätte sollen.
Möglicherweise wird ein besseres Ergebnis erzielt, wenn sie weniger Entscheidungen zu treffen hat.
Anstatt alle Attacken zu erlernen, könnte man diese auf wenige einschränken, oder ganz weglassen.

Performance-Probleme bei der Map-Erkennung, welche serverseitig passier, könnten behoben werden, indem
die Berechnungen clientseitig durchgeführt werden.